\documentclass[a4j,disablejfam,dvipdfmx,papersize,slide,uplatex,21pt]{jsarticle}
\usepackage[dvipdfmx]{graphicx}
\usepackage{parskip}
\usepackage{ascmac}
\usepackage{amsmath,amssymb}
\usepackage{amsthm}
    \makeatletter
    \renewenvironment{proof}[1][\proofname]{\par
        \pushQED{\qed}
        \normalfont
        \topsep6\p@\@plus6\p@ \trivlist
        \item[\hskip\labelsep{\bfseries #1}\@addpunct{\bfseries}]\ignorespaces
    }{%
        \popQED\endtrivlist\@endpefalse
    }
    \renewcommand{\proofname}{証明.}
    \makeatother
\usepackage{lastpage}
\usepackage{fancyhdr}
    \renewcommand{\headrulewidth}{0.0pt}
    \pagestyle{fancy}
    \lhead{}
    \chead{}
    \rhead{}
    \lfoot{}
    \cfoot{}
    \rfoot{\thepage{}/{}\pageref{LastPage}}
\usepackage[T1]{fontenc}
\usepackage{textcomp}
\usepackage[utf8]{inputenc}
\usepackage{bm}

\begin{document}

\title{「小林昭七 曲線と曲面の微分幾何」を読む\\
2021年4月16日}
\author{八幡 圭嗣}
\date{}
\maketitle

\section*{目次}
\begin{enumerate}
    \item 曲線の概念
    \item 曲線のパラメータ表示
    \item 曲率の導入
    \item 曲率の意味
    \item Gaussの表示
    \item 曲率による曲線の特徴付け
    \item 一般のパラメータによる曲率の表式
    \item 問題
\end{enumerate}



\section{曲線の概念}
本書は曲線という概念そのものの定義には深入りしていないが、
参考情報として曲線の定義に関するいくつかの立場に触れておく。

\newpage
\subsection*{参考:曲線の定義}
「$y=x^2$のグラフが曲線」とか「$F(x,y)=0$により定まる曲線」とかいうとき、
暗に点集合を指して曲線と呼んでいるものと考えられる。

一方、有界閉区間上の連続関数$\bm{p}(t) = (x(t), y(t))$自体を
(パラメータ付)曲線と呼ぶ立場もある\cite{kaiseki1}:

\begin{itembox}[l]{定義(パラメータ付曲線)\footnotemark[1]}
    $\bm{R}$の有界閉区間$I=[a, b]$上で定義された$\bm{R}^2$値連続関数$\bm{p}(t)$を
    パラメータ付曲線という。
\end{itembox}

この立場では$t$の増加する向きによって"曲線"に自然な向きが定まる。

\footnotetext[1]{
    より一般の多様体を扱う場合は定義域を開集合にしたり、
    逆関数の連続性を仮定したりする\cite{kaiseki2}\cite{manifold}。
}

\newpage
\subsection*{参考:パラメータ付曲線の同値}
2つのパラメータ付曲線に対し次のように同値関係を定め、
その同値類を曲線と呼ぶという立場がある\cite{kaiseki1}。
\begin{itembox}[l]{定義(パラメータ付曲線の同値)}
    2つのパラメータ付曲線$\bm{p}: I \rightarrow \mathbf{R}^2, \bm{q}: J \rightarrow \mathbf{R}^2$
    に対し次が成り立つとき、$\bm{p}, \bm{q}$は同値であるという:
    \begin{center}
        ある狭義単調増加な連続関数$\varphi: I \rightarrow \varphi(I) = J$が存在して
        $\bm{p} = \bm{q} \circ \varphi$
    \end{center}
\end{itembox}

\begin{itembox}[l]{定義(曲線)}
    上の同値関係により定まる同値類を曲線という。
    曲線$C$の元(すなわちパラメータ付曲線)を$C$のパラメータ表示という。
\end{itembox}

このとき曲線のパラメータ表示に像、始点、終点、弧長が存在すれば、それぞれはパラメータ表示のとり方によらず、
曲線のみによって定まる。


\newpage
\subsection*{パラメータ表示$\bm{p}(t)$に仮定すること}
本スライドではパラメータ表示$\bm{p}(t)$に対してさらに以下を仮定する。
\begin{itemize}
    \item $\bm{p}(t)$は2回微分可能とする。
    \item $\bm{p}(t)$は1対1であるとする。
    \item $\bm{p}(t)$は停留点を持たないとする。
\end{itemize}

\newpage
\subsection*{留意}
本書では「点集合としての曲線」「パラメータ付曲線」「パラメータ付曲線の同値類としての曲線」などを明確には区別していないため、
曲線という語の意味は文脈によって判断するしかない。




\section{曲線のパラメータ表示}
一般のパラメータ$t$によるパラメータ表示と、その特別な場合としての弧長$s$によるパラメータ表示について述べる。

\newpage
\subsection*{一般のパラメータ}
$\bm{p}(t)$を微分したものを
\begin{equation}
    \dot{\bm{p}}(t) = (\dot{x}(t), \dot{y}(t)),\quad
    \ddot{\bm{p}}(t) = (\ddot{x}(t), \ddot{y}(t))
\end{equation}
などと表す。
仮に$t$を時間とみなせば、物理的意味は
\begin{itemize}
    \item 位置ベクトル: $\bm{p}(t)$
    \item 速度ベクトル: $\bm{\dot{p}}(t)$
    \item 速さ: $|\dot{\bm{p}}(t)| = \sqrt{\dot{x}^2 + \dot{y}^2}$
    \item 加速度ベクトル: $\bm{\ddot{p}}(t)$
\end{itemize}
といえる。

\newpage
\subsection*{弧長}
パラメータが$0$から$t$まで動くときの弧長は
\begin{equation}
    s(t) = \int_0^t |\dot{\bm{p}}(t)| dt
\end{equation}
である。両辺を微分して
\begin{equation}
    \frac{ds}{dt}(t) = |\dot{\bm{p}}(t)|
\end{equation}
を得る。

\newpage
\subsection*{弧長パラメータ表示}
$\bm{p}(t)$は停留点を持たないから

$\phantom{\therefore}$ $\dfrac{ds}{dt} = |\dot{\bm{p}}| > 0$

$\therefore$ $s(t)$は狭義単調増加な連続関数

$\therefore$ $s(t)$は逆関数$t(s)$を持ち、$t(s)$も狭義単調増加な連続関数

\vspace{1em}

したがって、$\bm{p}(t)$が表す曲線は弧長によるパラメータ表示(省略して弧長パラメータ表示)$\bm{p}(s)$も可能である。
導関数は
\begin{equation}
    \bm{p}'(s) = (x'(s), y'(s))
\end{equation}
と表す。



\section{曲率の導入}
前頁で述べた弧長パラメータ表示$\bm{p}(s)$は、曲線に対して唯一つに定まる。
そこで次に$\bm{p}(s)$を用いて単位接ベクトル・単位法ベクトルを表し、それらによって曲率を定義する。

\newpage
\subsection*{単位接ベクトル$\bm{e}_1$}
ベクトル$\bm{e}_1(s)$を
\begin{equation}
    \bm{e}_1(s) = \bm{p}'(s)
\end{equation}
と定義する。
$\bm{e}_1(s_0)$は$\bm{p}(s)$で表される曲線の点$\bm{p}(s_0)$における単位接ベクトルである。

\newpage
\subsection*{単位法ベクトル$\bm{e}_2$}
ベクトル$\bm{e}_2(s)$を
\begin{center}
    各点で$\bm{e}_1(s)$に直交する単位ベクトルであって$\bm{e}_1$の左側にあるもの
\end{center}
と定義する。すなわち
\begin{equation}
    \bm{e}_2(s) = \bm{e}_1(s)
    \left[
    \begin{array}{cc}
        0 & 1 \\
        -1 & 0
    \end{array}
    \right]
\end{equation}
である。

\newpage
\subsection*{曲率}
$\bm{e}_1, \bm{e}_2$の関係を整理すると
\begin{alignat}{1}
    \bm{e}_1 \cdot \bm{e}_1 &= 1 \\
    \bm{e}_2 \cdot \bm{e}_2 &= 1 \\
    \bm{e}_1 \cdot \bm{e}_2 &= 0
\end{alignat}
である。$\bm{e}_1, \bm{e}_2$が正規直交基底をなすことに注意すれば、
ある実数値関数$\kappa(s)$が唯一つ存在して
\begin{equation}
    \bm{e}'_1 = \kappa \bm{e}_2,\quad
    \bm{e}'_2 = - \kappa \bm{e}_1
\end{equation}
すなわち
\begin{equation}
    \left[
        \begin{array}{c}
            \bm{e}'_1 \\
            \bm{e}'_2
        \end{array}
    \right]
    =
    \left[
        \begin{array}{cc}
            0 & \kappa \\
            -\kappa & 0
        \end{array}
    \right]
    \left[
        \begin{array}{c}
            \bm{e}_1 \\
            \bm{e}_2
        \end{array}
    \right]
\end{equation}
が成り立つ。この$\kappa(s)$を曲率(curvature)という。




\section{曲率の意味}
前節で定義された曲率が文字通り「曲がり具合」を表していることを、
いくつかの具体例によって確かめる。

\newpage
\subsection*{例2.1. 円の曲率}
パラメータ表示
\begin{equation}
    \bm{p}(t) = (r \cos t, r \sin t)
\end{equation}
で表される半径$r (> 0)$の円を考える。弧長は
\begin{alignat}{1}
    s(t) &= \int_0^t |\bm{p}'(t)| dt \\
        &= \int_0^t r dt \\
        &= rt
\end{alignat}
なので逆関数は$t(s) = \dfrac{s}{r}$であり、この円の弧長パラメータ表示は
\begin{equation}
    \bm{p}(s) = \left(r \cos \frac{s}{r}, r \sin \frac{s}{r}\right)
\end{equation}
となる。

\newpage
弧長パラメータ表示
\begin{equation}
    \bm{p}(s) = \left(r \cos \frac{s}{r}, r \sin \frac{s}{r}\right)
\end{equation}
から
\begin{alignat}{4}
    \bm{e}_1(s) &=  \bigl(&- &\sin \frac{s}{r},& &\cos \frac{s}{r} &&\bigr) \\
    \bm{e}_2(s) &=  \bigl(&- &\cos \frac{s}{r},& -&\sin \frac{s}{r} &&\bigr) \\
    \bm{e}'_1(s) &= \bigl(&- \frac{1}{r} &\cos \frac{s}{r},& - \frac{1}{r} &\sin \frac{s}{r} &&\bigr) \\
    \bm{e}'_2(s) &= \bigl(&\frac{1}{r} &\sin \frac{s}{r},& - \frac{1}{r} &\cos \frac{s}{r} &&\bigr)
\end{alignat}
を得る。したがって
\begin{equation}
    \bm{e}'_1 = \frac{1}{r} \bm{e}_2,\quad
    \bm{e}'_2 = -\frac{1}{r} \bm{e}_1
\end{equation}
ゆえに半径$r$の円の曲率は$\kappa = \dfrac{1}{r}$(定数)である。

\newpage
\subsection*{定理2.1. 直線の曲率}
\begin{itembox}[l]{定理2.1}
    弧長パラメータ表示$\bm{p}(s)$で表される曲線の曲率$\kappa(s)$について
    \begin{equation}
        \kappa = 0 \quad \Longleftrightarrow \quad \text{$\bm{p}(s)$が直線を表す}
    \end{equation}
\end{itembox}

ただし、「$\bm{p}(s)$が直線を表す」とは、
定数$\alpha, \beta, a, b, (\alpha, \beta) \neq (0, 0)$を用いて
\begin{alignat}{1}
    \bm{p}(s) = (\alpha, \beta) s + (a, b)
\end{alignat}
と表せることを指すものとする。

\newpage
\begin{proof}
    「$\kappa = 0 \Rightarrow \text{$\bm{p}(s)$が直線を表す}$」について

    $\kappa = 0$と仮定すると、$\bm{e}'_1 = \kappa \bm{e}_2 = \bm{0}$なので、
    各成分を$s$に関して積分すれば
    \begin{equation}
        \bm{e}_1 = \bm{p}' = (\alpha, \beta) \quad \text{($\alpha, \beta$は定数)}
    \end{equation}
    よって
    \begin{equation}
        \bm{p} = (\alpha, \beta) s + (a, b) \quad \text{($a, b$は定数)}
    \end{equation}
    と表せる。すなわち$\bm{p}(s)$は直線を表す。

    \newpage

    「$\kappa = 0 \Leftarrow \text{$\bm{p}(s)$が直線を表す}$」について

    $\bm{p}(s)$が直線を表すと仮定する。
    すなわち、定数$\alpha, \beta, a, b, (\alpha, \beta) \neq (0, 0)$を用いて
    \begin{alignat}{1}
        \bm{p} = (\alpha, \beta) s + (a, b)
    \end{alignat}
    と表せると仮定する。このとき$\bm{e}_1 = \bm{p}' = (\alpha, \beta)$より
    $\bm{e}'_1 = (0, 0), \bm{e}_2 = (-\beta, \alpha)$である。
    したがって、曲率の定義式$\bm{e}'_1 = \kappa \bm{e}_2$から
    $\kappa = 0$が従う。
\end{proof}

\newpage
\subsection*{2次近似}
$\bm{p}(s)$の点$s_0$における2次近似は
\begin{equation}
    \bm{p}(s_0) + \bm{p}'(s_0) (s - s_0) + \frac{1}{2} \bm{p}''(s_0) (s - s_0)^2
\end{equation}
すなわち
\begin{equation}
    \bm{p}(s_0) + \bm{e}_1(s_0) (s - s_0) + \frac{1}{2} \kappa(s_0) \bm{e}_2(s_0) (s - s_0)^2
\end{equation}
である。

\newpage
\subsection*{2次近似の例}
点$(0, 1)$を中心とする半径$1$の正の向きの円のパラメータ表示
\begin{equation}
    \bm{p}(t) = \left(\sin t, 1 - \cos t\right)
\end{equation}
の、原点における2次近似を考える。
例2.1(円の曲率)と同様に$\bm{e}_1, \bm{e}_2, \kappa$を求めると
\begin{equation}
    \bm{e}_1(0) = (1, 0),\quad
    \bm{e}_2(0) = (0, 1),\quad
    \kappa(0) = 1
\end{equation}
である。したがって、求める2次近似は
\begin{alignat}{1}
      &\bm{p}(0) + \bm{e}_1(0) s + \frac{1}{2} \kappa(0) \bm{e}_2(0) s^2 \\
    = &(0, 0) + (1, 0) s + \frac{1}{2} (0, 1) s^2 \\
    = &\left(s, \frac{1}{2} s^2\right)
\end{alignat}
である。




\section{Gaussの表示}
曲率の意味を説明するもうひとつの方法として、つぎにGaussの表示を考える。

\newpage
\subsection*{Gaussの表示}
「曲線$\bm{p}$の各点$\bm{p}(s)$に対して、原点からでる、$\bm{e}_2(s)$に平行なベクトルを対応させる。
点$\bm{p}(s)$からでているベクトル$\bm{e}_2(s)$も、原点からでている平行なベクトルも同じであるから、
同様に$\bm{e}_2(s)$と書くことにする。
曲線$\bm{p}$から単位円への対応$\bm{p}(s)\rightarrow\bm{e}_2(s)$をGaussの頭文字をとり
$g$と書くことにする。」

\newpage
点$\bm{p}$が$\bm{p}(s)$から$\bm{p}(s + \Delta s)$まで動くとき、
点$g(\bm{p})$は$\bm{e}_2(s)$から$\bm{e}_2(s + \Delta s)$まで動く。
\begin{alignat}{1}
    \bm{e}_2(s + \Delta s)
        &= \bm{e}_2(s) + \bm{e}_2'(s) \Delta s + o(\Delta s) \\
        &= \bm{e}_2(s) - \kappa(s) \bm{e}_1(s) \Delta s + o(\Delta s)
\end{alignat}
なので、点$\bm{e}_2$は速度$-\kappa\bm{e}_1$で単位円上を動く。

$g(\bm{p})$が定点であるのは曲率が常に$0$、すなわち直線のときで、そのときに限る。



\section{曲率による曲線の特徴付け}
\begin{alignat}{1}
    \text{曲線} &\rightarrow \text{曲率} \nonumber \\
    \text{曲線} &\xleftarrow{?} \text{曲率} \nonumber
\end{alignat}

\newpage
\subsection*{$\kappa$が一定の場合}
$\bm{p}(s)$で表される曲線の曲率を$\kappa(s)$とおく。
$\kappa$が$0$でない定数の場合を考える。
まず、「円の中心になるであろう点」$\bm{p} + \dfrac{1}{\kappa} \bm{e}_2$は定点である。実際
\begin{equation}
    \frac{d}{ds} \left( \bm{p} + \dfrac{1}{\kappa} \bm{e}_2 \right)
    = \bm{e}_1 + \frac{1}{\kappa} (- \kappa \bm{e}_1)
    = \bm{0}
\end{equation}
である。この定点から点$\bm{p}(s)$へのベクトルは
\begin{equation}
    \bm{p} - \left( \bm{p} + \dfrac{1}{\kappa} \bm{e}_2 \right)
    = - \frac{1}{\kappa} \bm{e}_2
\end{equation}
なので、$\bm{p}(s)$は半径$\dfrac{1}{|\kappa|}$の円を表し、
$\kappa > 0$ならば円は正の向き、$\kappa < 0$ならば円は負の向きである。

\newpage
\subsection*{定理2.2. 曲率による曲線の特徴付け}
\begin{itembox}[l]{定理2.2}
    $\bm{p}(s), \bar{\bm{p}}(s)$で表される曲線とその曲率$\kappa(s), \bar{\kappa}(s)$について
    \begin{equation}
        \kappa = \bar{\kappa}
        \quad \Longleftrightarrow \quad
        \text{回転と平行移動で$\bm{p}, \bar{\bm{p}}$が重なる}
    \end{equation}
\end{itembox}

\begin{proof}
    「$\Leftarrow$」について

    回転と平行移動で$\bm{p}, \bar{\bm{p}}$が重なると仮定する。
    すなわち、ある回転行列$A$と定ベクトル$\bm{b}$が存在して
    \begin{equation}
        \bar{\bm{p}}(s) = A \bm{p}(s) + \bm{b}
    \end{equation}
    と表せると仮定する。

    \newpage
    前頁の仮定
    \begin{equation}
        \bar{\bm{p}}(s) = A \bm{p}(s) + \bm{b}
    \end{equation}
    より、
    \begin{alignat}{1}
        \bar{\bm{e}}_1 &= \bar{\bm{p}}' \\
            &= A \bm{p}' \\
            &= A \bm{e}_1 \\
        \bar{\bm{e}}_2
            &=
                \left[
                    \begin{array}{cc}
                        0 & -1 \\
                        1 & 0
                    \end{array}
                \right]
                \bar{\bm{e}}_1 \\
            &=
                \left[
                    \begin{array}{cc}
                        0 & -1 \\
                        1 & 0
                    \end{array}
                \right]
                A \bm{e}_1 \\
            &=
                A \left[
                    \begin{array}{cc}
                        0 & -1 \\
                        1 & 0
                    \end{array}
                \right]
                \bm{e}_1 \\
            &= A \bm{e}_2
    \end{alignat}
    である。

    \newpage
    前頁の結果
    \begin{equation}
        \bar{\bm{e}}_1 = A \bm{e}_1, \quad
        \bar{\bm{e}}_2 = A \bm{e}_2
    \end{equation}
    より、
    \begin{equation}
        \bar{\bm{e}}_1' = A \bm{e}_1' = A (\kappa \bm{e}_2) = \kappa \bar{\bm{e}}_2
    \end{equation}
    である。
    このことと曲率の定義式$\bar{\bm{e}}_1' = \bar{\kappa} \bar{\bm{e}}_2$とを比較すれば
    $\kappa = \bar{\kappa}$を得る。

    \newpage
    「$\kappa = \bar{\kappa} \Rightarrow \text{回転と平行移動で$\bm{p}, \bar{\bm{p}}$が重なる}$」について

    $\kappa = \bar{\kappa}$を仮定する。
    まず適当な回転と平行移動で
    \begin{equation}
        \bm{p}(s_0) = \bar{\bm{p}}(s_0), \quad
        \bm{e}_1(s_0) = \bar{\bm{e}}_1(s_0)
    \end{equation}
    となるようにしておく。
    このとき$\bm{p} = \bar{\bm{p}}$であることを示そう。
    そのためには
    \begin{equation}
        \frac{d}{ds} (\bm{p} - \bar{\bm{p}}) = \bm{0}
    \end{equation}
    すなわち
    \begin{equation}
        \bm{e}_1 - \bar{\bm{e}}_1 = \bm{0}
    \end{equation}
    を示せばよい。

    \newpage
    各ベクトルの成分を
    \begin{alignat}{1}
              \bm{e}_1 &= (\xi_{11}, \xi_{12}) \\
              \bm{e}_2 &= (\xi_{21}, \xi_{22}) \\
        \bar{\bm{e}}_1 &= (\bar{\xi}_{11}, \bar{\xi}_{12}) \\
        \bar{\bm{e}}_2 &= (\bar{\xi}_{21}, \bar{\xi}_{22})
    \end{alignat}
    とおく。さらに
    \begin{alignat}{2}
        \bm{f}_1 &= \left[
            \begin{array}{c}
                \xi_{11} \\
                \xi_{21}
            \end{array}
        \right] \quad
        &
        \bm{f}_2 &= \left[
            \begin{array}{c}
                \xi_{12} \\
                \xi_{22}
            \end{array}
        \right] \\
        \bar{\bm{f}}_1 &= \left[
            \begin{array}{c}
                \bar{\xi}_{11} \\
                \bar{\xi}_{21}
            \end{array}
        \right] \quad
        &
        \bar{\bm{f}}_2 &= \left[
            \begin{array}{c}
                \bar{\xi}_{12} \\
                \bar{\xi}_{22}
            \end{array}
        \right]
    \end{alignat}
    とおく。
    このとき、行列
    \begin{equation}
        X = \left[
            \begin{array}{c}
                \bm{e}_1 \\
                \bm{e}_2
            \end{array}
        \right]
        = [
            \begin{array}{cc}
                \bm{f}_1 & \bm{f}_2
            \end{array}]
    \end{equation}
    は直交行列であるから、$\bm{f}_1, \bm{f}_2$は正規直交基底である($\bar{\bm{f}}_1, \bar{\bm{f}}_2$も同様)。

    \newpage
    さて、今示したいことは
    \begin{equation}
        \bm{e}_1 - \bar{\bm{e}}_1 = \bm{0}
    \end{equation}
    であった。そのためには
    \begin{equation}
        |\bm{f}_1 - \bar{\bm{f}}_1|^2 = 0, \quad
        |\bm{f}_2 - \bar{\bm{f}}_2|^2 = 0
    \end{equation}
    を示せばよい。

    \newpage
    まず$|\bm{f}_1 - \bar{\bm{f}}_1|^2 = 0$を示す。ここで
    \begin{alignat}{1}
        |\bm{f}_1 - \bar{\bm{f}}_1|^2
            &= (\xi_{11} - \bar{\xi}_{11})^2 + (\xi_{21} - \bar{\xi}_{21})^2 \\
            &= \xi_{11}^2 + \xi_{21}^2 + \bar{\xi}_{11}^2 + \bar{\xi}_{21}^2
                - 2 (\xi_{11} \bar{\xi}_{11} + \xi_{21} \bar{\xi}_{21}) \\
            &= 2 - 2 (\xi_{11} \bar{\xi}_{11} + \xi_{21} \bar{\xi}_{21})
    \end{alignat}
    なので
    \begin{equation}
        \xi_{11} \bar{\xi}_{11} + \xi_{21} \bar{\xi}_{21} = 1
    \end{equation}
    を示せばよいが、$s = s_0$のときは
    \begin{alignat}{1}
        \xi_{11}(s_0) \bar{\xi}_{11}(s_0) + \xi_{21}(s_0) \bar{\xi}_{21}(s_0)
            &= \xi_{11}(s_0)^2 + \xi_{21}(s_0)^2 \\
            &= |\bm{f}_1|^2 \\
            &= 1
    \end{alignat}
    が成り立っているから、結局
    \begin{equation}
        \frac{d}{ds} (\xi_{11} \bar{\xi}_{11} + \xi_{21} \bar{\xi}_{21}) = 0
    \end{equation}
    を示せばよい。

    \newpage
    前頁の考察で、$|\bm{f}_1 - \bar{\bm{f}}_1|^2 = 0$を示すには
    \begin{equation}
        \frac{d}{ds} (\xi_{11} \bar{\xi}_{11} + \xi_{21} \bar{\xi}_{21}) = 0
    \end{equation}
    を示せばよいことがわかった。
    実際に$s$について微分してみると
    \begin{alignat}{1}
        \frac{d}{ds} (\xi_{11} \bar{\xi}_{11} + \xi_{21} \bar{\xi}_{21})
            &= \xi_{11}' \bar{\xi}_{11} + \xi_{11} \bar{\xi}_{11}'
                + \xi_{21}' \bar{\xi}_{21} + \xi_{21} \bar{\xi}_{21}' \\
            &= \kappa \xi_{21} \bar{\xi}_{11} + \kappa \xi_{11} \bar{\xi}_{21}
                - \kappa \xi_{11} \bar{\xi}_{21} - \kappa \xi_{21} \bar{\xi}_{11} \\
            &= 0
    \end{alignat}
    なので、示したかった$|\bm{f}_1 - \bar{\bm{f}}_1|^2 = 0$がいえた。
    同様にして$|\bm{f}_2 - \bar{\bm{f}}_2|^2 = 0$も示せる。

    以上より
    「$\kappa = \bar{\kappa} \Rightarrow \text{回転と平行移動で$\bm{p}, \bar{\bm{p}}$が重なる}$」がいえた。
\end{proof}




\section{一般のパラメータによる曲率の表式}
弧長$s$によるパラメータ表示を具体的に求めることは一般には難しい。
そこで、一般のパラメータ$t$による曲率の表式を考える。

\newpage
\subsection*{例2.2. 楕円の曲率}
楕円のパラメータ表示
\begin{equation}
    x = a \cos t,\quad y = b \sin t \quad (a > b > 0)
\end{equation}
から曲率を求める。
\begin{alignat}{2}
    &&s &= \int_0^t \sqrt{\dot{x}^2 + \dot{y}^2} dt \\
    &\therefore& \quad \frac{ds}{dt} &= \sqrt{\dot{x}^2 + \dot{y}^2} \\
        &&&= \sqrt{a^2 \sin^2 t + b^2 \cos^2 t} \\
    &\therefore& \quad \frac{dt}{ds} &= \frac{1}{\sqrt{a^2 \sin^2 t + b^2 \cos^2 t}}
\end{alignat}

\newpage
また
\begin{alignat}{2}
    &&\bm{e}_1 &= \bm{p}' \\
        &&&= \dot{\bm{p}} \frac{dt}{ds} \\
        &&&= (-a \sin t, b \cos t) \frac{dt}{ds} \\
    &\therefore& \quad \bm{e}_2 &= (-b \cos t, -a \sin t) \frac{dt}{ds}
\end{alignat}
である。

\newpage
前頁で
\begin{alignat}{1}
    \bm{e}_1 &= (-a \sin t, b \cos t) \frac{dt}{ds} \\
    \bm{e}_2 &= (-b \cos t, -a \sin t) \frac{dt}{ds}
\end{alignat}
を得た。ここで
\begin{alignat}{1}
    \bm{e}_1' &= \dot{\bm{e}}_1 \frac{dt}{ds} \\
        &= \left(
            \frac{-ab^2 \cos t}{(a^2 \sin^2 t + b^2 \cos^2 t)^{3/2}},
            \frac{-a^2b \sin t}{(a^2 \sin^2 t + b^2 \cos^2 t)^{3/2}}
        \right) \frac{dt}{ds} \\
        &= \frac{ab}{(a^2 \sin^2 t + b^2 \cos^2 t)^{3/2}}
            (-b \cos t, -a \sin t) \frac{dt}{ds} \\
        &= \frac{ab}{(a^2 \sin^2 t + b^2 \cos^2 t)^{3/2}}
            \bm{e}_2
\end{alignat}
なので
\begin{equation}
    \kappa = \frac{ab}{(a^2 \sin^2 t + b^2 \cos^2 t)^{3/2}}
\end{equation}
である。




\section{問題}

\newpage
\subsection*{一般の曲線に対する$\kappa(t)$}
\begin{itembox}[l]{問2.1}
    パラメータ表示$\bm{p}(t) = (x(t), y(t))$で表される曲線の曲率は
    \begin{equation}
        \kappa(t) = \frac{\dot{x}(t) \ddot{y}(t) - \ddot{x}(t) \dot{y}(t)}{(\dot{x}(t)^2 + \dot{y}(t)^2)^{3/2}}
    \end{equation}
    である。
\end{itembox}

\newpage
\begin{proof}
    まず
    \begin{alignat}{2}
        && \frac{dt}{ds} &= \frac{1}{\frac{ds}{dt}} \\
            &&&= \frac{1}{|\dot{\bm{p}}|} \\
            &&&= \frac{1}{\sqrt{\dot{x}^2 + \dot{y}^2}}
    \end{alignat}
    である。また
    \begin{alignat}{2}
        &&\bm{e}_1 &= \bm{p}' \\
            &&&= \dot{\bm{p}} \frac{dt}{ds} \\
            &&&= (\dot{x}, \dot{y}) \frac{dt}{ds} \\
        &\therefore& \quad \bm{e}_2 &= (-\dot{y}, \dot{x}) \frac{dt}{ds}
    \end{alignat}
    である。

    \newpage
    ここで
    \begin{alignat}{1}
        \bm{e}_1' &= \dot{\bm{e}}_1 \frac{dt}{ds} \\
            &= \left(
                \frac{-(\dot{x}\ddot{y} - \ddot{x}\dot{y}) \dot{y}}{(\dot{x}^2 + \dot{y}^2)^{3/2}},
                \frac{ (\dot{x}\ddot{y} - \ddot{x}\dot{y}) \dot{x}}{(\dot{x}^2 + \dot{y}^2)^{3/2}}
            \right) \frac{dt}{ds} \\
            &= \frac{\dot{x}\ddot{y} - \ddot{x}\dot{y}}{(\dot{x}^2 + \dot{y}^2)^{3/2}}
            \left(
                -\dot{y},
                 \dot{x}
            \right) \frac{dt}{ds} \\
            &= \frac{\dot{x}\ddot{y} - \ddot{x}\dot{y}}{(\dot{x}^2 + \dot{y}^2)^{3/2}}
                \bm{e}_2
    \end{alignat}
    なので
    \begin{equation}
        \kappa = \frac{\dot{x}\ddot{y} - \ddot{x}\dot{y}}{(\dot{x}^2 + \dot{y}^2)^{3/2}}
    \end{equation}
    である。
\end{proof}

\newpage
\subsection*{極座標で表される曲線の曲率}
\begin{itembox}[l]{問2.2}
    パラメータ表示$\bm{p}(\theta) = (r \cos\theta, r\sin\theta), r = F(\theta)$で表される曲線の曲率は
    \begin{equation}
        \kappa =
            \frac
                {r^2 + 2 \left( \dfrac{dr}{d\theta} \right)^2 - r \dfrac{d^2r}{d\theta^2}}
                {\left\{ r^2 + \left( \dfrac{dr}{d\theta} \right)^2 \right\}^{3/2}}
    \end{equation}
    である。
\end{itembox}

\newpage
\begin{proof}
    \begin{alignat}{1}
        \frac{dx}{d\theta} &= \frac{dr}{d\theta} \cos\theta - r \sin\theta \\
        \frac{dy}{d\theta} &= \frac{dr}{d\theta} \sin\theta - r \cos\theta \\
        \frac{d^2x}{d\theta^2} &= \frac{d^2r}{d\theta^2} \cos\theta - 2 \frac{dr}{d\theta} \sin\theta - r \cos\theta \\
        \frac{d^2y}{d\theta^2} &= \frac{d^2r}{d\theta^2} \sin\theta - 2 \frac{dr}{d\theta} \cos\theta - r \sin\theta
    \end{alignat}
    より
    \begin{alignat}{1}
        \left(\frac{dx}{d\theta}\right)^2 + \left(\frac{dy}{d\theta}\right)^2
            &= r^2 + \left(\frac{dr}{d\theta}\right)^2 \\
        \frac{dx}{d\theta} \frac{d^2y}{d\theta^2} - \frac{d^2x}{d\theta^2}\frac{dy}{d\theta}
            &= r^2 + 2\left(\frac{dr}{d\theta}\right)^2 - r \frac{d^2r}{d\theta^2}
    \end{alignat}
    である。

    \newpage
    前頁の結果
    \begin{alignat}{1}
        \left(\frac{dx}{d\theta}\right)^2 + \left(\frac{dy}{d\theta}\right)^2
            &= r^2 + \left(\frac{dr}{d\theta}\right)^2 \\
        \frac{dx}{d\theta} \frac{d^2y}{d\theta^2} - \frac{d^2x}{d\theta^2}\frac{dy}{d\theta}
            &= r^2 + 2\left(\frac{dr}{d\theta}\right)^2 - r \frac{d^2r}{d\theta^2}
    \end{alignat}
    を問2.1の式に代入して
    \begin{equation}
        \kappa =
            \frac
                {r^2 + 2 \left( \dfrac{dr}{d\theta} \right)^2 - r \dfrac{d^2r}{d\theta^2}}
                {\left\{ r^2 + \left( \dfrac{dr}{d\theta} \right)^2 \right\}^{3/2}}
    \end{equation}
    を得る。
\end{proof}

\newpage
\subsection*{双曲線の曲率}
\begin{itembox}[l]{問2.3}
    パラメータ表示$\bm{p}(t) = (\cosh t, \sinh t)$で表される曲線の曲率は
    \begin{equation}
        \kappa = - \frac{1}{(\cosh 2t)^{3/2}}
    \end{equation}
    である。
\end{itembox}

\newpage
\begin{proof}
    各成分を微分したものは
    \begin{alignat}{1}
        \dot{x} &= \sinh t \\
        \dot{y} &= \cosh t \\
        \ddot{x} &= \cosh t \\
        \ddot{y} &= \sinh t
    \end{alignat}
    なので、問2.1の式より
    \begin{alignat}{1}
        \kappa &= \frac{\sinh^2 t - \cosh^2 t}{(\sinh^2 t + \cosh^2 t)^{3/2}} \\
            &= - \frac{1}{(\sinh^2 t + \cosh^2 t)^{3/2}} \\
            &= - \frac{1}{(\cosh 2t)^{3/2}}
    \end{alignat}
    である。
\end{proof}

\newpage
\subsection*{懸垂線の弧長}
\begin{itembox}[l]{問2.4}
    $a > 0$とする。パラメータ表示$\bm{p}(t) = \left(t, a \cosh \dfrac{t}{a} \right)$で表される曲線の
    $0 \le t \le x$における弧長は
    \begin{equation}
        s = a \sinh \frac{x}{a}
    \end{equation}
    であり、弧長によるパラメータ表示は
    \begin{equation}
        \bm{p}(s) = \left(
            a \sinh^{-1} \frac{s}{a},
            \sqrt{a^2 + s^2}
        \right)
    \end{equation}
    である。
\end{itembox}

\newpage
\begin{proof}
    弧長は
    \begin{alignat}{1}
        s &= \int_0^x |\dot{\bm{p}}(t)| dt \\
            &= \int_0^x \cosh t dt \\
            &= a \sinh \frac{x}{a}
    \end{alignat}
    である。よって
    \begin{alignat}{1}
        x &= a \sinh^{-1} \frac{s}{a} \\
        y &= a \sqrt{1 + \sinh^2 \frac{x}{a}} \\
            &= a \sqrt{1 + \left(\frac{s}{a}\right)^2} \\
            &= \sqrt{a^2 + s^2}
    \end{alignat}
    である。
\end{proof}




\begin{thebibliography}{99}
    \bibitem{kaiseki1} 杉浦光夫、解析入門I、東京大学出版会
    \bibitem{kaiseki2} 杉浦光夫、解析入門II、東京大学出版会
    \bibitem{senkei} 齋藤正彦、線型代数入門、東京大学出版会
    \bibitem{manifold} 松本幸夫、多様体の基礎、東京大学出版会
\end{thebibliography}

\end{document}
